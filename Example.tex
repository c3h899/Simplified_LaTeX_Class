% Content of Example is NOT author's original work.

% === Document Header === %
\documentclass{StdTemplate}

% Base Document Attributes
\title{Document Title}
\author{John Doe}
\date{\today}

% Customization of Document Header
\HeaderLeft{ECE 5341} % Left Justified Portion of Document Header
\HeaderRight{\theauthor} % Right Justified Portion of Document Header
\HeaderRight{\thedate} % Right Justified Portion of Document Header
\UpdateHeaddings % Populates header with the contents of \HeaderLeft and \HeaderRight

\begin{document}
\TitlePage
% === Document Body === %
\text{This is the text command.}

\begin{abstract}
    This is the document abstract. % === Document Header === %
\documentclass{StdTemplate}

% Base Document Attributes
\title{Document Title}
\author{John Doe}
\date{\today}

% Customization of Document Header
\HeaderLeft{ECE 5341} % Left Justified Portion of Document Header
\HeaderRight{\theauthor} % Right Justified Portion of Document Header
\HeaderRight{\thedate} % Right Justified Portion of Document Header
\UpdateHeaddings % Populates header with the contents of \HeaderLeft and \HeaderRight

\begin{document}
\TitlePage
% === Document Body === %
\text{This is the text command.}

\begin{abstract}
    This is the document abstract. The pack­age adds an op­tional ar­gu­ment to the enu­mer­ate en­vi­ron­ment which de­ter­mines the style in which the counter is printed. The en­u­mitem pack­age su­per­sedes—it pro­vides the same fa­cil­i­ties in a well-struc­tured way.
\end{abstract}

\begin{lstlisting}[language=Python, caption=Now It's Captioned, label=label1]
print("Hello World")
\end{lstlisting}

\begin{align}
f(x) &= a x^2 + b x + c\label{eq1}\\
g(x) &= a^\prime (x - x_0)^2 + c^\prime \text{ This is plain-text}\label{eq2}
\end{align}

When I want to reference the second equation one can call it as \ref{eq2}, and reference equation one as \ref{eq1}.

\begin{align*}
f(x) &= a x^2 + b x + c\\
g(x) &= a^\prime (x - x_0)^2 + c^\prime \text{ This is plain-text}
\end{align*}


\section{Section 1}

\begin{enumerate}
\item 1st level
    \begin{enumerate}
    \item 2nd level
        \begin{enumerate}
        \item 3rd level
            \begin{enumerate}
            \item 4th level
            \end{enumerate}
        \end{enumerate}
    \end{enumerate}
\end{enumerate}

\begin{table}[h]
\begin{tabular}{c|c }
A & B \\
C & D 
\end{tabular}
\caption{This is a table}
\end{table}

\section{Section 2}

\begin{enumerate}
\item 1st level
    \begin{enumerate}
    \item 2nd level
        \begin{enumerate}
        \item 3rd level
            \begin{enumerate}
            \item 4th level
            \end{enumerate}
        \end{enumerate}
    \end{enumerate}
\end{enumerate}

\begin{enumerate}[resume]
\item 1st level
    \begin{enumerate}
    \item 2nd level
        \begin{enumerate}
        \item 3rd level
            \begin{enumerate}
            \item 4th level
            \end{enumerate}
        \end{enumerate}
    \end{enumerate}
\end{enumerate}

\begin{algorithm}[H]
\caption{How to write algorithms}
 \KwData{this text}
 \KwResult{how to write algorithm with \LaTeX2e }
 initialization\;
 \While{not at end of this document}{
  read current\;
  \eIf{understand}{
   go to next section\;
   current section becomes this one\;
   }{
   go back to the beginning of current section\;
  }
 }
 
\end{algorithm}
%\begin{lstlisting}[language=SQL]%
%SELECT retards FROM moba_games WHERE griefing > 100 AND ban IS true%
%\end{lstlisting}%


\end{document}
\end{abstract}

\begin{lstlisting}[language=Python, caption=Now It's Captioned, label=label1]
print("Hello World")
\end{lstlisting}

\begin{align}
f(x) &= a x^2 + b x + c\label{eq1}\\
g(x) &= a^\prime (x - x_0)^2 + c^\prime \text{ This is plain-text}\label{eq2}
\end{align}

When I want to reference the second equation one can call it as \ref{eq2}, and reference equation one as \ref{eq1}.

\begin{align*}
f(x) &= a x^2 + b x + c\\
g(x) &= a^\prime (x - x_0)^2 + c^\prime \text{ This is plain-text}
\end{align*}


\section{Section 1}

\begin{enumerate}
\item 1st level
    \begin{enumerate}
    \item 2nd level
        \begin{enumerate}
        \item 3rd level
            \begin{enumerate}
            \item 4th level
            \end{enumerate}
        \end{enumerate}
    \end{enumerate}
\end{enumerate}

\begin{table}[h]
\begin{tabular}{c|c }
A & B \\
C & D 
\end{tabular}
\caption{This is a table}
\end{table}

\section{Section 2}

\begin{enumerate}
\item 1st level
    \begin{enumerate}
    \item 2nd level
        \begin{enumerate}
        \item 3rd level
            \begin{enumerate}
            \item 4th level
            \end{enumerate}
        \end{enumerate}
    \end{enumerate}
\end{enumerate}

\begin{enumerate}[resume]
\item 1st level
    \begin{enumerate}
    \item 2nd level
        \begin{enumerate}
        \item 3rd level
            \begin{enumerate}
            \item 4th level
            \end{enumerate}
        \end{enumerate}
    \end{enumerate}
\end{enumerate}

\begin{algorithm}[H]
\caption{How to write algorithms}
 \KwData{this text}
 \KwResult{how to write algorithm with \LaTeX2e }
 initialization\;
 \While{not at end of this document}{
  read current\;
  \eIf{understand}{
   go to next section\;
   current section becomes this one\;
   }{
   go back to the beginning of current section\;
  }
 }
 
\end{algorithm}
%\begin{lstlisting}[language=SQL]%
%SELECT retards FROM moba_games WHERE griefing > 100 AND ban IS true%
%\end{lstlisting}%


\end{document}
